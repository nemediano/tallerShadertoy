\section{GLSL: Lenguaje para escribir shaders}

\begin{frame}[fragile]{Tipos primitivos}
Esta \alert{no es} una lista exhaustiva, ante la duda consulta la \href{https://www.khronos.org/opengl/wiki/Data_Type_(GLSL)}{referencia}.
\begin{itemize}
    \item Tipos de escalares: \mintinline{glsl}{float}, \mintinline{glsl}{int}, \mintinline{glsl}{bool}.
    \item Vectores de $n \in \{2, 3, 4\}$ dimensiones: \mintinline{glsl}{vec3}, \mintinline{glsl}{ivec2}, \mintinline{glsl}{bvec4}.
    \begin{itemize}
        \item Nótese que cuando el tipo subyacente es \mintinline{glsl}{float} no se requiere el prefijo.
    \end{itemize}
    \item Los operadores aritméticos entre vectores se aplican por componente.
    \begin{itemize}
        \item Requieren que los operandos sean del mismo tamaño
    \end{itemize}
    \item Matrices de $n\times m$ dimensiones donde $n, m \in \{2, 3, 4\}$ dimensiones: \mintinline{glsl}{mat2x3}, \mintinline{glsl}{mat3x4}.
    \begin{itemize}
        \item Todas las matrices son de tipo \mintinline{glsl}{float}
        \item Las matrices cuadradas se pueden abreviar: \mintinline{glsl}{mat2}
        \item Las operaciones entre matrices, son las esperadas del álgebra lineal
    \end{itemize}
    \item Las multiplicaciones: \say{matriz por matriz}, \say{matriz por vector}, \say{escalar por vector} y  \say{escalar por matriz} son las definida en álgebra lineal.
\end{itemize}

\end{frame}

\begin{frame}[fragile]{Swizzling}
\begin{itemize}
    \item Los tipos vectoriales tienen accesores y mutadores a sus componentes individuales.
    \item Los accesores tienen tres sintaxis equivalentes: $(x, y, z, w), (r, g, b, a), (s, t, p, q)$.
\end{itemize}
Esto define el llamado Swizzling:
\begin{listing}
\begin{minted}{glsl}
vec2 someVec;
vec4 otherVec = someVec.xyxx;
vec3 thirdVec = otherVec.zyy;
vec4 fourthVec;
// Tambien funciona en l-values:
fourthVec.wzyx = vec4(1.0, 2.0, 3.0, 4.0); // Reverses the order.
fourthVec.zx = vec2(3.0, 5.0); // Sets the 3rd component to 3 and the 1st component to 5
\end{minted}
\end{listing}
\end{frame}

\begin{frame}[fragile]{Constructores de vectores}
Se pueden construir a partir de:
\begin{itemize}
    \item Vectores de mayor dimensión (los componentes extra son ignorados).
    \item Una combinación de escalares y de vectores de menor dimensión.
    \item De manera abreviada, especificando un solo escalar que se repite.
\end{itemize}
\begin{listing}
\begin{minted}{glsl}
vec4(vec2(10.0, 11.0), 1.0, 3.5) == vec4(10.0, vec2(11.0, 1.0), 3.5);
vec3(vec4(1.0, 2.0, 3.0, 4.0)) == vec3(1.0, 2.0, 3.0);
vec4(vec3(1.0, 2.0, 3.0)); // error. Not enough components.
vec2(vec3(1.0, 2.0, 3.0)); // OK
vec3(1.0) // Abreviation to say: vec3(1.0, 1.0, 1.0);
\end{minted}
\end{listing}
\end{frame}


\begin{frame}[fragile]{Constructores de matrices}
\begin{itemize}
    \item Se construyen por columnas.
    \item Se pueden construir a partir de matrices de menor o igual dimensión.
    \item O de manera abreviada especificando un solo escalar que llena la diagonal.
\end{itemize}
\begin{listing}
\begin{minted}{glsl}
mat2(
  float, float,   // first column
  float, float);  // second column

mat4(
  vec4,           // first column
  vec4,           // second column
  vec4,           // third column
  vec4);          // fourth column

mat3 diagMatrix = mat3(5.0);  // Diagonal matrix with 5.0 on the diagonal.
mat4 otherMatrix = mat4(diagMatrix); // The last element on the diagonal is 1.0
\end{minted}
\end{listing}
\end{frame}

\begin{frame}{Built-in functions}
Además de las funciones habituales esperadas, algunas funciones interesantes:
\begin{table}[htb]
  \begin{center}
    \begin{tabular}{l | l }
      \mintinline{glsl}{vec3 mix(vec3 x, vec3 y, float a)} & Interpolación lineal \\
      \mintinline{glsl}{vec3 step(float edge, vec3 x)} & Función escalón \\
      \mintinline{glsl}{mat4 inverse(mat4 M)} & Matriz inversa \\
      \mintinline{glsl}{float dot(vec3 x, vec3 y)} & Producto punto \\
      \mintinline{glsl}{vec3 cross(vec3 x, vec3 y)} & Producto cruz \\
      \mintinline{glsl}{float mod(float x, float k)} & El residuo de dividir $x$ entre $k$ \\
      \mintinline{glsl}{vec3 clamp(vec3 x, vec3 min, vec3 max)} & Limitar entre dos valores \\
      \mintinline{glsl}{float length(vec3 x)} & Norma de un vector \\
      \mintinline{glsl}{vec3 normalize(vec3 x)} & Vector normalizado \\
      \mintinline{glsl}{float floor(vec3 x)} & Entero $f$ mas cercano a $x$, tal que $f \leq x$ \\
    \end{tabular}
  \end{center}
\end{table}
Todas las funciones tienen las sobrecargas vectoriales, cuando corresponde.
\end{frame}
